\documentclass[a4j,12pt]{jarticle}
\usepackage[dvipdfmx]{graphicx}
\usepackage[dvipdfmx]{hyperref}
%
% ---- 本文中でプログラムを掲載する際にキャプションを「リスト1」のようにする設定
%
% http://en.wikibooks.org/wiki/LaTeX/Floats,_Figures_and_Captions#Custom_floats
% 本文中でリストと表現されているところは
%
% \begin{program}..\end{program}を使います。
%
%  \begin{program}\centering
%  \begin{verbatim}
%
%  #define COM1_PORT (0x3f8)
%  #define COM1_LSR (COM1_PORT + 0)
%  #define COM1_RBR (COM1_PORT + 5)
%  unsigned char read_reg_byte(unsigned short port) {
%    unsigned char val;
%    asm volatile("inb %1, %0" : "=a"(val) : "Nd"(port));
%    return val;
%  }
%  \end{verbatim}
%  \caption{I/O マップド I/O での read\_reg\_byte() 関数およびレジスタの宣言}
%  \end{program}
%
% TODO: programをリストに変更する
%
\usepackage{float}
% 例では次のようになっているが...
%\newfloat{program}{thp}{lop}
\newfloat{program}{thp}{lop}
% ------------------------------------------------------------------------------
